\chapter{Materiais e Métodos}

O presente capítulo tem o objetivo de descrever os materiais utilizados, as ferramentas empregadas e o método aplicado para realizar a análise dos protocolos abordados, tanto na elaboração e condução do experimento pelo qual se fundamenta este trabalho, como na captura, tratamento e análise dos dados coletados a partir do mesmo.  

\section{O Experimento}

O experimento  

\section{Os dados}

Os dados que serão transmitidos pela placa, que será tratado como dados de envio pelo restante do capítulo, foram obtidos previamente a partir de uma estação meteorológica durante o período de dezembro de 2016 à dezembro de 2017 e podem ser acessados através do link [http://raphael.rdsoares.com/projeto/tcc/dados]. 

Os dados de envio contemplam um total de 9168 registros e todos estão no formato JSON. Estes registros contém, além do timestamp que registra o momento exato em que cada registro foi capturado, informações obtidas durante 1 hora naquela estação meteorológica, como por exemplo: radiação, temperatura mínima e máxima, umidade mínima e máxima, velocidade do vento, pressão mínima e máxima, precipitação, entre outros. 

\section{Escolha da placa}

Para a realização do experimento foi necessário a aquisição de uma placa para servir como central de recepção dos dados enviados 

- Escolha da placa
- Escolha das APIs de cada protocolo
- Definição dos fatores (e dos níveis de cada fator)
- Definição do tipo do experimento
- Condução do experimento