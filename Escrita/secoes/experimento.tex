Este capítulo destina-se a apresenta como foi elaborado e realizado o experimento. Nele serão abordados os seguintes tópicos: a descrição do experimento, a escolha dos dados trafegados no experimento, a escolha da placa p/ transferência dos dados, os fatores utilizados, as API's dos protocolos e a arquitetura geral do experimento.

\subsection{Descrição do Experimento}

Para este trabalho, foi elaborado um experimento que consistiu em escolher um conjunto fixo de dados e, utilizando uma placa IoT, enviá-los através de uma rede local para um notebook fazendo uso de cada um dos protocolos estudados neste trabalho. No intuito de deixar o experimento mais próximo da realidade, foi utilizado o emulador de rede NETEM para simular uma rede de baixa qualidade. 

A escolha do conjunto de dados foi feita visando obter informações capturadas por sensores reais. Para tal, foi utilizado dados capturados de hora em hora de todos os dias do o ano de 2017 de uma estação meteorológica. Estes dados foram utilizados no formato JSON, onde cada objeto equivale a uma hora de um dia do ano. Cada um desses objetos contém as seguintes informações: data, hora, timestamp, precipitação, radiação, rajada de vento, velocidade do vento, direção do vento, temperatura mínima, máxima e naquele instante, umidade mínima, máxima e naquele instante, pressão mínima, máxima e naquele instante, ponto do orvalho mínimo, máximo e naquele instante.

Para a escolha da placa IoT responsável pelo envio dos dados foi levado em conta fatores como facilidade de aquisição, preço, API's disponíveis para os protocolos escolhidos, facilidade de implementação e manutenção. As placas analisadas foram a ESP-8266 e a Raspberry Pi. Devido a facilidade de aquisição e de implementação a placa escolhida para o experimento foi a Raspberry Pi.
