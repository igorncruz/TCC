\chapter{Introdução}

\section{Apresentação}

No primeiro semestre de 2018 a humanidade alcançou a marca de 17 bilhões de dispositivos conectados \cite{url_lueth2018_7BiIoT}. O avanço tecnológico e a computação ubíqua chegaram a tal ponto que hoje a expressão "navegar na internet" ou "surfar na rede" está defasada, "mergulhar na rede" talvez seja o termo mais apropriado. A sociedade moderna está tão cercada por e imersa na tecnologia no seu cotidiano que chega a ser difícil encontrar, no ambiente urbano, uma pessoa sem acesso à Internet.

Dentre estes 17 bilhões há um nicho de 7 bilhões que vem crescendo de forma quase exponencial nos últimos anos e com perspectiva de crescer muito mais, são os dispositivos de Internet das Coisas (Internet of Things, ou IoT). Dispositivos IoT são basicamente objetos que possuem alguma tecnologia embarcada, sensores e conexão com Internet, tornando-os capazes de coletar, transmitir e, em alguns casos, processar dados \cite{artigo_santos2016_IoTDaTeoriaAPratica}. Hoje já existe uma enorme variedade deles no mercado, tais como lâmpadas, geladeiras, micro-ondas inteligentes, e outros tantos chamados "dispositivos inteligentes". A expectativa é que eles estejam cada vez mais inseridos no cotidiano das pessoas, pois é esperado um crescimento de 10 bilhões de dispositivos IoT em 2020 e 22 bilhões em 2025. Em paralelo a esta expansão, há uma prospecção de crescimento do mercado global de IoT de 37\% em 2018, saltando de U\$ 110 bilhões para U\$ 151 bilhões e com perspectiva de chegar a U\$ 1,56 trilhão em 2025 \cite{url_lueth2018_7BiIoT}.

Atualmente já existem diversos exemplos de como estes dispositivos podem ajudar a sociedade em suas atividades diárias, tanto no individual quanto no coletivo. Na medicina, no monitoramento dos sinais vitais e condições de saúde dos pacientes em hospitais e em casas de idosos. Nas residências, nas medições de consumo de energia remotamente e monitoramento de câmeras de segurança à distância. No transporte, com sensores de proximidade nos carros ajudando a evitar colisões e controle de tráfego nas vias ajudando a melhorar o trânsito. Na agricultura, com sensores de temperatura, umidade e luminosidade nas fazendas e plantações ajudando a otimizar a produção de comida \cite{artigo_mancini2017_iotAplicacoes}.

Apesar do progresso científico estar avançando consideravelmente neste setor, por ser um campo relativamente recente, a tecnologia para realizar de forma satisfatória e eficaz a comunicação entre estes objetos, também chamada comunicação Maquina-para-Maquina (em inglês: Machine to Machine, ou M2M) ainda é muito incipiente. É bastante comum haver falha no envio de informações entre estes dispositivos, seja por problemas de conectividade ou de rede, seja por quebra ou desligamento inesperado de um equipamento, seja por manutenção incorreta ou escassa, ou por algum outro fator externo [PROCURAR UMA REFERENCIA!!!]. O fato é que a troca de informações entre dispositivos IoT é um fator crucial para o avanço e solidificação dessa tecnologia e precisa ser estudado com mais atenção. 

Com base nisto, este trabalho se propõe a analisar e comparar 6 protocolos de comunicação mais comumente utilizados em dispositivos IoT para transferência de dados na camada de aplicação. São eles: HTTP, MQTT, CoAP, XMPP, AMQP e STOMP. Para tal foi realizado um experimento seguindo os moldes de design de experimentos em que foi utilizada uma placa Raspberry Pi como dispositivo IoT cuja responsabilidade foi o envio de dados, utilizando cada um dos 6 protocolos mencionados, capturados previamente de uma estação meteorológica, e um notebook como servidor de aplicação que foi responsável pelo recebimento destes dados enviados. Os pacotes de dados transferidos da placa para o notebook foram capturados, tratados analisados e comparados. Através desta comparação foi possível chegar ao objetivo do trabalho.


\section{Motivação e Justificativa}

Como mencionado anteriormente, a área de Internet das Coisas vem num crescimento constante e, a medida que os avanços tecnológicos vão progredindo, os custos de produção desses dispositivos vão barateando e se popularizando cada vez mais.  

Apesar desses avanços, a comparação entre protocolos M2M não foi alvo de muitos estudos e, dentre estes, a extensa maioria deles se focou nos 2 mais utilizados, o MQTT e o CoAP.

Além de concentrar uma boa quantidade de protocolos de comunicação IoT, cujas arquiteturas diferem bastante entre si, este estudo inclui também, como um dos protocolos de comparação, o HTTP, que é o protocolo mais tradicional e amplamente utilizado hoje em dia na Internet.


\section{Objetivos}

O presente trabalho tem por objetivo principal avaliar e comparar a eficiência dos protocolos de comunicação nos dispositivos de Internet das Coisas. Através desta comparação pretende-se atender os seguintes objetivos específicos:

\begin{itemize}
    \item Avaliar os protocolos quanto ao atraso no tempo de envio e de recebimento dos pacotes;
    \item Avaliar os protocolos quanto a quantidade de dados trafegada por segundo;
    \item Avaliar os protocolos no quesito de quantidade de pacotes extras gerados durante a comunicação;
    \item Avaliar os protocolos quanto a quantidade e porcentagem de pacotes perdidos com relação a quantidade total de pacotes enviados;
\end{itemize}


\section{Organização do Trabalho}

O presente trabalho está organizado em seis capítulos. O capítulo atual que visa introduzir o contexto geral em que problema se encontra inserido, além da motivação, justificativa e objetivos que nortearam sua execução. O demais capítulos se encontram divididos da seguinte maneira:

\textbf{Capítulo 2}: Trata da fundamentação teórica, onde são apresentados os conceitos envolvidos no trabalho, bem como a arquitetura e o \textit{modus operandi} de cada protocolo.

\textbf{Capítulo 3}: São discutidos alguns trabalhos relacionados cujo objeto de estudo também é a comparação entre protocolos de comunicação.

\textbf{Capítulo 4}: Apresenta a metodologia e arquitetura empregada no experimento, juntamente com os métodos de captura dos dados para análise.

\textbf{Capítulo 5}: Expõe e discute os resultado a partir análise e interpretação dos dados obtidos.

\textbf{Capítulo 6}: Traz as considerações finais do trabalho, junto com suas contribuições, objetivos alcançados e dificuldades encontradas, e propõe possíveis trabalhos futuros.