\chapter{Trabalhos relacionados}

Neste capítulo são apresentados alguns trabalhos similares cujos objetivos se enquadram na questão de análise de protocolos na comunicação M2M (Machine-to-Machine), tanto na camada de redes, quanto em outras camadas, como a de enlace.

\cite{chen2016performance} avaliaram a aplicabilidade da IoT na área médica. Em seu trabalhos, eles propuseram o seguinte cenário: um paciente utiliza diversos tipos sensores pelo corpo, que captam várias informações sobre o mesmo, tais como nível de O², pressão arterial, atividade cerebral, atividade cardíaca, atividade muscular e sensor de inércia. Em seguida esses sensores enviam esses dados para uma placa central, que ele chama de Patient Gateway, e essa placa centraliza as informações e envia via Wireless para um servidor. Esse servidor então fica responsável por processar essas informações e enviar para o smartphone do médico. 
O estudo deles focou em 4 tipos de protocolos IoT para o envio dos dados do Patient Gateway para o servidor comparados através de 3 parâmetros de avaliação de redes. Os protocolo foram CoAP, MQTT, DDS e um 4º protocolo criado por eles que envia dados no formado JSON através do protocolo UDP. Já os parâmetros de avaliação foram consumo de banda, latência e taxa de perda de pacotes. 
Através desta sua análise, eles chegaram a conclusão que protocolos que utilizam TCP superam os protocolos UDP diante de uma rede de baixa qualidade devido a sua perda de pacotes quase nula, porém o consumo de banda e a latência aumentam. Também concluíram que o protocolo DDS é mais efetivo que o protocolo MQTT devido a sua latência ter sido menor. Já os protocolos UDP, por possuírem uma taxa de perda de pacotes imprevisível, se tornam interessantes para aplicações onde o envio e a captação de dados não sejam críticos para o sistema, o que não é o caso da área médica. 

\cite{deCaro2013comparison} realizaram a comparação tanto quantitativa quanto qualitativa entre os protocolos MQTT e CoAP no que tange aos aplicativos mobile devido ao fato de considerarem os smartphones atuais como sensores sofisticados ambulantes, pois os mais atuais contam com GPS, câmera, microfone, giroscópio, acelerômetro, bússola, sensor de proximidade, de luminosidade, de luminosidade e de temperatura.
Em sua comparação foi utilizado o Smartphone Samsung Galaxy S Plus rodando Android 4.2.2 conectado via WiFi a um servidor, um notebook rodando Windows 7.
Nos testes foram avaliados o consumo de banda, latência (através do RTT) e taxa de perda de pacotes dos protocolos utilizando tanto mensagens com confiabilidade quanto sem confiabilidade.
Eles concluíram que o CoAP apresentou melhores resultado nos quesitos de consumo de banda e RTT, o que o torna mais eficiente quando o objetivo redução na utilização dos recursos do dispositivo em questão e do consumo de dados da rede. Já o MQTT, por conta da sua confiabilidade, se torna mais indicado para aplicações mais sofisticadas, que requisitam controles de congestionamento e garantia que os dados irão chegar ao destinatário. Porém, em aplicações que não requerem uma alta frequência na transmissão dos dados, a taxa de confiabilidade de ambos os protocolos se mantém praticamente a mesma.

\cite{bandyopadhyay2013lightweight} também realizaram uma análise entre os protocolos CoAP e MQTT. No trabalho deles foram considerados os aspectos consumo de energia, consumo de banda, confiabilidade, entre outros. É importante salientar que neste trabalho eles abordam as possíveis arquiteturas utilizando o CoAP, tanto no seu aspect request-response, quanto no modo resource-observer, onde seu funcionamento se assemelha ao MQTT. Neste trabalho eles utilizaram apenas computadores rodando o sistema operacional Ubuntu tanto para o Cliente quando para o Servidor, utilizaram o software WANEM para fazer o controle da rede e simular uma rede de baixa qualidade e por último o software Wireshark para analisar o tráfego na rede. O experimento deles consistiu na análise dos dados trafegados na rede e do consumo de energia em kwh com taxa de perda de pacotes em 0\% e em 20\% nas duas arquiteturas citados acima do CoAP e com o MQTT e eles concluíram que o CoAP é mais eficiente tanto no consumo de energia quanto no consumo de banda.

\cite{thangavel2014commonMiddleware} realizaram a análise entre os protocolos CoAP e MQTT através de um intermediário customizado. Este intermediário consistiu numa API que eles criaram que serviu de ponte entre a captação dos dados pelos sensores e o envio desses dados pelos protocolos. A partir dessa API eles avaliaram a influência de diversos parâmetros na performance desses protocolos. A performance foi medida em termos de delay e total de dados (bytes) transferidos por mensagem. Eles consideraram esse total de dados transferidos como o indicador do consumo de banda. Já o delay foi mensurado pela diferença do tempo em que o dado foi recebido pelo servidor e que foi enviado.
No experimento eles fizeram uso de um notebook que serviu como servidor tanto pro CoAP quanto para o MQTT, uma placa BeagleBoard-xM que foi o responsável tanto pela captação dos dados dos sensores (onde foi implementado a API customizada) e envio destes para o servidor, quanto pela recepção dos dados processados. Por fim um Netbook rodando o software Wanem, cujo propósito foi emular uma rede de baixa qualidade.
Ao final do experimento eles chegaram a conclusão que a performance dos protocolos variam de acordo com a qualidade da rede, pois o MQTT experimentou um delay menor que o CoAP quando o a perda de pacotes era pequena porém seu delay foi superior ao CoAP quando a perda de pacotes era alta. Também concluíram que, para tamanhos de mensagens pequenos e perda de pacotes igual ou menor que 25\%, o CoAP gera menos trafego extra na rede do que o MQTT para obter uma conexão confiável.

\cite{gloria2017comparison} focaram seu estudo na camada de enlace. Eles avaliaram 4 tipos de protocolos, sendo 2 deles com fio e 2 sem fio, foram eles: I$^{2}$C, RS232, ZigBee e LoRa. Para esta análise foram utilizaram 3 placas: Arduino, Raspberry Pi e ESP12 para avaliar se a diferença entre as placas ou a distância em que elas estão uma da outra interfere no delay, na taxa de dados trafegadas e na eficiência dos protocolos. Eles concluíram que o RS232 é a melhor escolha quando se trata de um protocolo com fio e LoRa por sua vez é a melhor nos protocolos sem fio. Isso se deu por conta da baixa complexidade e baixo custo envolvidos e pelo fato de não gerarem interferência com redes WiFi. Com relação as placas, a que obteve o melhor resultado foi a Arduino conectada com a ESP12 tanto por conta da confiabilidade quanto por conta da Raspberry Pi, que por ser uma placa que roda em cima de um sistema operacional, possui características extras por vezes não desejadas, tais como navegador de internet, sistemas de arquivos entre outras, que acabam requisitando um consumo de energia mais elevado.


