\setlength{\absparsep}{18pt} % ajusta o espaçamento dos parágrafos do resumo
\begin{resumo}
 
No primeiro semestre de 2018 chegamos a marca de 17 bilhões de dispositivos conectados à Internet em todo o mundo. Desses 17, 10 bilhões são equipamentos como smartphones, tablets e notebooks, e 7 bilhões são os chamados IoT (Internet of Things, ou Internet das Coisas), ou seja, são "coisas" que estão conectadas à rede, mas que essa conexão não está entre suas principais funções. Isso implica dizer que hoje, há um número gigantesco de máquinas conversando com maquinas...

Baseado nisto, o presente trabalho propõe avaliar os 6 protocolos de comunicação mais comuns, HTTP, MQTT, CoAP, AMQP, STOMP e XMPP, utilizados na camada de aplicação para o envio de dados entre dispositivos, e elencar os mais rápidos, robustos e confiáveis. Para tal, foi elaborado um experimento utilizando uma placa Raspberry Pi, servindo como dispositivo IoT, um notebook e um switch. O experimento consistiu no envio, utilizando os protocolos mencionados anteriormente, de uma certa quantidade de dados reais, extraídos de uma estação meteorológica, partindo da placa Rasp para o notebook. Para deixar o experimento o mais próximo possível da realidade, foi utilizado um simulador de redes para emular uma rede de baixa qualidade. Os pacotes de dados enviados foram capturados e foi realizada uma análise estatística para comparar 3 fatores: velocidade, latência e perda de pacotes. A partir do resultado dessa análise foi possível elencar os melhores protocolos e uma sugestão dos seus possíveis usos.

 \textbf{Palavras-chave}: Internet das Coisas, Protocolos, Camada de Aplicação, Raspberry Pi, HTTP, MQTT, CoAP, AMQP, STOMP, XMPP.
\end{resumo}

